\name{plot.pco}
\alias{plot.pco}
\alias{points.pco}
\alias{plotid.pco}
\alias{hilight.pco}
\alias{surf.pco}
\alias{jsurf.pco}

\title{Plotting Routines For Principal Coordinates Ordinations}
\description{A set of routines for plotting, highlighting points,
or adding fitted surfaces to PCOs.}
\usage{
\method{plot}{pco}(x, ax = 1, ay = 2, col = 1, title = "", pch = 1, \dots)
\method{points}{pco}(x, which, ax = 1, ay = 2, col = 2, pch = 1, cex = 1, \dots)
\method{plotid}{pco}(ord, ids = seq(1:nrow(ord\$points)), ax = 1, ay = 2, 
    col = 1, \dots)
\method{hilight}{pco}(ord, factor, ax = 1, ay = 2, \dots)
\method{surf}{pco}(ord, var, ax = 1, ay = 2, col = 2, labcex = 0.8, 
    family = gaussian)
\method{jsurf}{pco}(ord, var, ax = 1, ay = 2, col = 2, labcex = 0.8, 
    family = gaussian)
}

\arguments{
    \item{x}{an object of class \sQuote{pco}}
    \item{ax}{the dimension to use for the X axis}
    \item{ay}{the dimension to use for the Y axis}
    \item{title}{a title for the plot}
    \item{which}{a logical variable to specify points to be highlighted}
    \item{ord}{an object of class \sQuote{pco}}
    \item{factor}{a factor or integer vector}
    \item{var}{a variable to be surfaced}
    \item{family}{controls the link function passed to \sQuote{gam}: 
          one of \sQuote{gaussian}, \sQuote{binomial}, or \sQuote{poisson}}
    \item{ids}{identifier labels for samples.  Defaults to 1:n}
}

\details{
Function \sQuote{plot} produces a scatterplot of sample
scores for the specified axes, erasing or over-plotting
on the current graphic device.  Axes dimensions are controlled to produce
a graph with the correct aspect ratio.
Functions \sQuote{points}, \sQuote{plotid}, \sQuote{surf}, and 
\sQuote{jsurf} add detail to an existing plot.  The axes specified must
match the underlying plot exactly.  

Function \sQuote{plotid} identifies 
and labels samples (optionally with values from a third vector) 
in the PCO, and requires interaction with the mouse: 
left button identifies, right button exits}

Function \sQuote{points} is passed a logical 
vector to identify a set of samples by color of glyph.  It can be used
to identify a single set meeting almost any criterion that can be
stated as a logical expression.

Function \sQuote{hilight} is passed a factor vector or integer vector,
and identifies identifies factor values by color and glyph.

Functions \sQuote{surf} and \sQuote{jsurf} calculate and plot fitted 
surfaces for logical or quantitative
variables; \sQuote{jsurf} jitters the coordinates to avoid problems with 
identical values in fitting the surface.  The functions are simply
wrappers for the \code{\link[mgcv]{gam}}} function, and send the 
sample coordinates and the \sQuote{var} variable 
to \sQuote{gam} to model.  The coordinates and the fitted values from 
\sQuote{gam} are then passed to \code{\link[akima]{interp}}, which plots the 
contours.  The default link function for fitting the GAMs is sQuote{gaussian}, 
suitable for unbounded continuous variables.  For logical variables you 
should specify \sQuote{family = binomial} to get a logistic GAM, and 
for integer counts you should specify \sQuote{family = poisson} to get 
a Poisson GAM.
}

\value{Function \sQuote{plotid} returns a vector of row numbers of
identified plots}

\author{
David W. Roberts
\email{droberts@montana.edu}
\url{http://ecology.msu.montana.edu/labdsv}
}

\seealso{
\code{\link[labdsv]{points}}, \code{\link[labdsv]{plotid}},
\code{\link[labdsv]{hilight}},
\code{\link[labdsv]{surf}}, and \code{\link[labdsv]{jsurf}}
}

\examples{
data(bryceveg)
data(brycesite)
dis.bc <- dsvdis(bryceveg,'bray/curtis')
pco.1 <- pco(dis.bc,5)
plot(pco.1)
points(pco.1,brycesite$elev>8000)
surf(pco.1,brycesite$elev)
\dontrun{plotid(pco.1,ids=row.names(bryceveg))}
}
